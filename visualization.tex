\section{Map Visualization}
The Map Visualization part of the Visualization module is a part of the system that allows a user to visualize data through geographic maps upon choosing the field in the data. The module uses different libraries: 
\begin{itemize}
	\item \textbf{Google Maps API}
	This library is being used for visualizing data through geographical maps. Images that are rendered on visualization are supported by Google Maps API library. 
	\item \textbf{D3.js}
	This library implements the numerical and non-numerical 2-variable visualization. After the user selected two fields that are numerical and non-numerical(metadata should be uploaded), the library will render donut marker that represents per household. Since non-numerical fields generate randomize color and Google Maps API doesn't support rendering donuts. D3.js is used. 
	
	\item \textbf{Angular Range-Slider}
	This library implements the two-handler slider that simulates future status per household. Instead of having one-handler slider that has fix minimum and adjustable maximum, two-handler slider is implemented. It allows the user flexibility on having an adjustable minimum and maximum and also visualizes the household inside the adjusted minimum and maximum. 
	
	\item \textbf{Chroma.js}
	This library is needed to generate colors for non-numerical items. Colors can range from orange, yellow, blue, indigo and violet. Red and green colors are not allowed since it might confuse the user non-numerical to poverty indicator formula.
\end{itemize}
\subsection{Household Level Visualization}
Household level visualization is one of the feature of this module. Using the Google Maps API and D3.js, the user can visualize one to two fields selected by the user. By determining the color appropriate for a certain household because of his/her status, marker is rendered with a custom image as its icon. Rendering a custom icon is important because it lessen the stress of the system if thousand or millions of the data are needed to be visualized. Numerical and non-numerical visualization are not handled by Google Maps API because of the randomize color used by non-numerical field. D3.js handles that situation. D3.js handles the visualization of the numerical and non-numerical 2 variable visualization.
\subsection{Barangay Level Visualization}
Barangay level visualization uses Google Maps API also as its main library. The user can visualize all barangays or selected barangays. By converting the shape files into GEOJSON format and uses the Google Maps API for map setting and style or coloring of the polygons. 

\clearpage